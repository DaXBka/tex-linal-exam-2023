\task{3}
\begin{condition}
    В евклидовом пространстве $\R^3$ со стандартным скалярным произведением даны векторы:
    \[
        u_1 = (-1, 1, 2), \;
        u_2 = (1, 1, -1), \;
        u_3 = (1, -1, 0)
    \]
    Обозначим через $v_1, v_2, v_3$ ортогональные проекции вектора $v = (3, -5, 1)$ на подпространства $u_1^\bot, u_2^\bot, u_3^\bot$ соответственно. Найдите объем параллелепипеда, натянутого на векторы $v_1, v_2, v_3$.
\end{condition}

Для нахождения векторов $v_1, v_2, v_3$ в явном виде воспользуемся таким фактом:
\begin{theorem}
    \[
        \forall \; u, v \in \E \colon \pr_u v = \ort_{u^\bot} v, \; \ort_u v = \pr_{u^\bot} v
    \]
\end{theorem}

Выпишем векторы, заданные условием:
\begin{itemize}
    \item $v_1 = \pr_{u_1^\bot} v = \ort_{u_1} v = v - \pr_{u_1} v = v - \d{(v, u_1)}{(u_1, u_1)} u_1 = v + u_1 = (2, -4, 3)$

    \item $v_2 = \pr_{u_2^\bot} v = \ort_{u_2} v = v - \pr_{u_2} v = v - \d{(v, u_2)}{(u_2, u_2)} u_2 = v + u_2 = (4, -4, 0)$

    \item $v_3 = \pr_{u_3^\bot} v = \ort_{u_3} v = v - \pr_{u_3} v = v - \d{(v, u_3)}{(u_3, u_3)} u_3 = v - 4 u_3 = (-1, -1, 1)$
\end{itemize}

Существует много различных способов найти объем параллелепипеда, я воспользуюсь вычислением через матрицу Грама:
\begin{theorem}
    \[
        \vol P(v_1, v_2, v_3)^2 = \det G(v_1, v_2, v_3)
    \]
\end{theorem}

Соберём матрицу Грама явно:
\[
    G := G(v_1, v_2, v_3)
    =
    \begin{pmatrix}
        (v_1, v_1) & (v_1, v_2) & (v_1, v_3) \\
        (v_2, v_1) & (v_2, v_2) & (v_2, v_3) \\
        (v_3, v_1) & (v_3, v_2) & (v_3, v_3)
    \end{pmatrix}
    =
    \begin{pmatrix}
        29 & 24 & 5 \\
        24 & 32 & 0 \\
        5  & 0  & 3
    \end{pmatrix}
\]

Найдём определитель для $G$:
\[
    \det G =
    29 \cdot 32 \cdot 3 + 0 + 0 - 5 \cdot 5 \cdot 32 - 0 - 24 \cdot 24 \cdot 3 = 2784 - 800 - 1728 = 256
\]

Тогда $\vol P(v_1, v_2, v_3)^2 = \det G = 256$, получаем $\vol P(v_1, v_2, v_3) = \sqrt{256} = 16$.

\answer{3}
Объем параллелепипеда, натянутого на векторы $v_1, v_2, v_3$, равен 16.
