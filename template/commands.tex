%%%%% Множества и системы %%%%%
\newcommand{\E}{\mathbb{E}}
\newcommand{\D}{\mathbb{D}}
\renewcommand{\C}{\mathbb{C}}
\newcommand{\R}{\mathbb{R}}
\newcommand{\N}{\mathbb{N}}
\newcommand{\Z}{\mathbb{Z}}
\newcommand{\Q}{\mathbb{Q}}
\newcommand{\F}{\mathbb{F}}
\newcommand{\e}{\mathbb{e}}
\renewcommand{\f}{\mathbb{f}}

%%%%% Математические обозначения %%%%%
\renewcommand{\tr}{tr}
\DeclareMathOperator{\vht}{ht}
\DeclareMathOperator{\rk}{rk}
\DeclareMathOperator{\pr}{pr}
\DeclareMathOperator{\ort}{ort}
\DeclareMathOperator{\Mat}{Mat}
\DeclareMathOperator{\vol}{vol}
\DeclareMathOperator{\Spec}{Spec}
\DeclareMathOperator{\id}{id}
\DeclareMathOperator{\Ker}{Ker}
\DeclareMathOperator{\M}{M}

%%%%% Полезности %%%%%
\renewcommand{\r}{\right}
\renewcommand{\l}{\left}
\renewcommand{\d}{\frac}
\renewcommand{\leq}{\leqslant}
\renewcommand{\geq}{\geqslant}
\newcommand{\Sum}[2]{\overset{#2}{\underset{#1}{\sum}}}
\newcommand{\Prod}[2]{\overset{#2}{\underset{#1}{\prod}}}
\newcommand{\Lim}[2]{\lim\limits_{#1 \to #2}}
\newcommand{\task}[1] {\noindent \section*{Задача №#1:}}
\newcommand{\answer}[1] {\noindent\makebox[\linewidth]{\rule{\textwidth}{0.4pt}} \subsection*{Ответ №#1:}}


%%%%% Полезные блоки %%%%%

% Расширенная матрица
\newenvironment{amatrix}[2]
{\left(\begin{array}{@{}*{#1}{c}|*{#2}{c}@{}}}
        {\end{array}\right)}

% Блок с условием
\newenvironment{condition}
{\begin{tcolorbox}[colback = green!10!white, colframe = green!75!black]}
        {\end{tcolorbox}}

% Блок с теоремой
\newenvironment{theorem}
{\begin{tcolorbox}[colback = blue!5!white, colframe = blue!75!black, opacityframe=0.2,opacityback=0.8]}
        {\end{tcolorbox}}
